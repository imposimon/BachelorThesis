\chapter{简单映射的发现}
\label{chap02}

简单映射的发现主要分为两个步骤:基于逆向工程规则的元素分类和基于虚拟文档的简单
映射发现。

\section{元素类型分类}
\label{2:1}
文献\cite{14}在对关系数据库逆向工程的研究中,将关系分为4种不相交的类型:强实体
类型关系(strong entity relation,简称SER)、弱实体类型关系(weak entity
relation,简称WER)、常规关系类型关系(regular relationship relation,简称RRR)
、以及特殊关系类型关系(specific relationship relation,简称SRR)。而属性也可
简单地分为两类:外键类型属性(FKA)和非外键类型属性(NFKA)。

具体判别规则如下:如果一个关系的所有主键均不是外键,则该关系属于强实体类型关系
,例如图\ref{fig:db}中的\emph{student};如果关系的所有主键均为外键,并且主键数目
大于2,则属于常规关系类型关系,例如图\ref{fig:db}中的\emph{takes\_courese};
而对于弱实体类型关系和特殊关系类型关系,则需要推理和人工参与。如果一个关系是强
实体类型关系的子关系,也就是说该关系中的部分主键作为外键,并且仅只想某一个强实
体类型关系,则这个关系属于弱实体类型关系;对于特殊关系类型关系的判断与之相似。
如果上述自动推理不能判断,则需要人工参与识别。另外,有时还需要存储的实例数据加
以验证。对于外键类型属性和非外键类型属性的判别则相对简单,在此不再赘述。

一般而言,一个实体类型关系可以映射到本体中的一个类,而一个关系类型关系可以映射
到本体中的一个对象属性。例如,图\ref{fig:db}所示的关系\emph{student}可以与类
\emph{Student}映射;而关系\emph{takes\_course}可以与对象属性\emph{takesCourse}
映射。类似地,如果一个属性是外键类型属性,则它可以映射到本体中的一个对象属性,
而一个非外键属性则既可以映射到本体中的一个数据类型属性,也可以映射到一个对象属
性。这里需要注意一种例外情况,如果一个关系是关系类型关系,则它所有的作为主键
并且同时作为外键的属性不需要参与映射过程,如图\ref{fig:db}中的关系
\emph{takes\_course}的属性\emph{\underline{sid}}和\emph{\underline{cid}}
不必包含在映射结果中,否则会导致重复。

根据上述启发式规则,本文把关系数据库模式和本体中的元素对应地划分为4类:

\begin{itemize}
\item{$\{\{SER\} \cup \{WER\} \} \times \{C\}$}
\item{$\{\{RRR\} \cup \{SRR\} \} \times \{P_o\}$}
\item{$\{FKA\} \times \{P_o\}$}
\item{$\{NFKA\} \times \{\{P_o\} \cup \{P_d\}\}$}
\end{itemize}

此外,还设计一些预处理步骤来协调关系数据模式和本体的不同特性。首先,一个关系类
型关系允许映射到本体中的两个具有互反关系(owl:inverseOf)的对象属性。其次
,关系数据库模式中的多元关系($\ge3$)应被具体化(reify)为多个二元关系,因为OWL
本体只能表达一元或二元关系。另外,还对关系数据库和本体中常见的数据类型做抽象,
将主要数据类型归为整型、浮点型、字符型、日期型和布尔型。请注意,这里的预处理步
骤并不是完备的,但是它们涵盖了较多的实际情况。

\section{基于虚拟文档的简单映射发现}
\label{2:2}
受文献\cite{15}的启发,本文考虑引入关系数据库模式和本体间的结构特征来体现它们对
应的语义信息。比如关系数据库中的引用完整性约束连接了两个关系,而OWL本体可以表示
为RDF图的形式,该图结构包含了重要的语义信息。

本文为关系数据库模式和本体中的每个元素构建虚拟文档(virtual document),记作
$vdoc()$,以捕获它们所含的语义信息。一个虚拟文档形如一个加权的单词集合,不仅包含
元素自身的自然语言描述,还引入周围相邻元素的自然语言描述。

为了度量一个单词的重要程度,本文采用了TF-IDF模型\cite{32}。TF-IDF用于表示一个单
词对于一个文档集合中一个文档的重要程度。其中,词频TF(term frequency)指某单词
在某文档中出现的频率。对于文档$d$中的特定单词$t$来说,其词频计算公式为:

\begin{equation}
tf_{t,d} = \frac{n_{t,d}}{\sum_{k}n_{k,d}}
\end{equation}

其中$n_{t,d}$表示单词$t$在文档$d$中的出现次数,分母$\sum_{k}n_{k,d}$
表示文档$d$中所有单词的出现次数之和。

逆向文档频率IDF(inverse document frequency)反映了一个单词在一个文档集合中的
普遍重要性。对于某单词$t$,其逆向文档频率计算公式为:

\begin{equation}
idf_t = log \frac{|D|}{|\{j:t \in d_j\}|} 
\end{equation}

其中$|D|$表示文档集合$D$中的文档总数,$|\{j:t \in d_j\}|$表示包含单词$t$
的文档数量。词频和逆向文档频率的乘积即为TF-IDF值:

\begin{equation}
tf\textrm{-}idf_{t,d} = tf_{t,d} \times idf_t
\end{equation}

虚拟文档中采用单词的TF-IDF值作为其权值,单词的权值反映其重要程度。
每个虚拟文档可以被看成是一个TF-IDF模型中的向量。本文为属于不同类别的元素构建
不同的虚拟文档。

对于关系数据库模式中的一个关系$R$,如果它属于实体类型关系,则它的虚拟文档是它
自身的自然语言描述;而如果属于关系类型的关系,则它的虚拟文档不仅是其自身的自然
语言描述,还包括它所引用的关系的描述。形式化描述如下:

\begin{equation}
vdoc(R)=
\begin{cases}
desc(R)
\qquad \qquad \qquad \qquad \qquad \qquad \qquad
 R \in \{\emph{SER}\} \cup \{\emph{WER}\}
\\
desc(R)+ \alpha \ast \sum\limits_{A^\prime \in ref(A) \atop A \in pk(R)}
desc(rel(A^\prime))
\qquad \ R \in \{\emph{RRR}\} \cup \{\emph{SRR}\}
\end{cases}
\end{equation}

对于关系数据库模式中的一个属性A,除了其自身的自然语言描述外(含所属的关系),如
果它是外键类型属性,则还进一步考虑它引用的属性所属的关系的描述;而如果是非外键
类型属性,则补充考虑它的数据类型,具体公式如下:

\begin{equation}
vdoc(A)=
\begin{cases}
desc(A)+ \alpha \ast (desc(rel(A)) + \sum\limits_{A^\prime \in ref(A)}
desc(rel(A^\prime)))
\ \ A \in \{\emph{FKA}\}
\\
desc(A)+ \alpha \ast desc(rel(A)) + \beta \ast desc(type(A))
\qquad \ A \in \{\emph{NFKA}\}
\end{cases}
\end{equation}

对于本体中的一个类$C$,它的虚拟文档就是其自身的自然语言描述,写作如下公式:

\begin{equation}
vdoc(C) = desc(C)
\end{equation}

对于本体中的一个属性$P$,不仅考虑其自身的自然语言描述,还考虑它的定义与和值域
。请注意,如果一个属性是数据类型属性,此时它的值域为它的数据类型。公式如下:

\begin{equation}
vdoc(P)=
\begin{cases}
desc(P) + \alpha \ast (
	\sum\limits_{C \in d(P)} desc(C) +
	\sum\limits_{C \in r(P)} desc(C))
\qquad \ P \in \{P_o\}
\\
desc(P) + \alpha \ast \sum\limits_{C \in d(P)} desc(C)
	+ \beta \ast desc(r(P))
\qquad \quad  P \in \{P_d\}
\end{cases}
\end{equation}

为简单起见,$desc()$仅返回元素的本地名(local name)作为它自身的自然语言描述。
$\alpha$和$\beta$为赋予不同邻居的权重,是两个固定的有理数,属于值域$[0,1]$
。这两个参数的值应当依据实际应用而设定。根据我们的经验,$\alpha$一般比$\beta$
略大。

为了展示虚拟文档的构建过程,请看图\ref{fig:db}中的关系类型关系
\emph{takes\_course}。它自身的自然语言描述是\{``takes'',``course''\},
且存在两个邻居关系\emph{course}和\emph{student}。\emph{takes\_course}的虚拟
文档为$vdoc(takes\_course)=
\{\mbox{``take''},(1+\alpha)\ast\mbox{``course''},
\alpha\ast\mbox{``student''}
\}$。

通过计算元素之间的相似度获得简单映射。任意两个元素之间的相似度通过计算它们单词
向量之间的余弦值获得,这两个向量分别对应两个虚拟文档。相似度的计算公式如下:

\begin{equation}
sim(vdoc_i,vdoc_j)=cos(\overrightarrow{N_i},\overrightarrow{N_j})=\frac
	{\sum_{k=1}^{D}n_{ik}n_{jk}}
	{\sqrt{ \sum_{k=1}^{D}n_{ik}^{2} 
		\sum_{k=1}^{D}n_{jk}^{2}
	}}
\end{equation}

其中$D$是向量空间的维度,$n_{ik}$和$n_{jk}$是向量中的元素。如果两个虚拟文档没
有共同的单词则其相似度为0,而如果两个虚拟文档完全相同则相似度为1。
