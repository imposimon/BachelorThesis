\chapter{相关工作}
\label{chap05}

现有工作从多个方面研究了关系数据库模式和本体间的映射问题,例如设计映射系统的框架
、提出具体映射算法以及定义映射结果的语法语义。有关介绍请参见研究综述\cite{4,7}。

DartGrid是一个中医药领域的数据集成系统\cite{25},其中的DartMapping模块提供了一个
可视化的工具,帮助领域专家手工定义关系数据库模式与本体间的映射。OntoMat-Reverse
\cite{26}使用逆向工程规则和基于编辑距离的文本相似度计算方法,半自动地发现关系数
据库中表/列和本体中类/属性间的映射,与它类似的工作还有RONTO\cite{27}、Marson
\cite{24}等。OntoGrate框架\cite{28}首先将每个关系数据库模式转化为对应的DB本体,
然后借助记录链接和多关系数据挖掘等技术,高度自动化地发现DB本体和其他语义网本体之
间的映射。此外OntoGrate还设计了一种称为Web-PDDL的映射语言,将针对本体的查询转换
为SQL查询。类似地,StdTrip\cite{29}也是将关系数据库模式转化为本体后再实施本体映
射。MapOnto\cite{10}使用树状结构作为数据库模式和本体的中间转换模型,基于预先发现
的简单映射,在两个中间模型上迭代地传播这些映射,最终发现关系数据库模式中元素和
本体元素之间的多对多映射,并以Horn字句的形式表达。Marson则基于具有分类特性的关系
数据库列(例如性别的取值有``男''和``女''),运用决策数算法构造一类具有包含关系的
复杂映射,可以转化为基于视图的查询。

对比上述研究工作,在简单映射发现方面,本文提出了基于虚拟文档的方法,通过考察元素周
围的各种邻居元素的自然语言描述,显著提高了映射结果的效果。在复杂映射学习方面,除
MapOnto和Marson以外,其它工作还很少考虑复杂映射。MapOnto仅利用属性的定义域/值域
的兼容性构建一种句法层次上的复杂映射,Marson则是基于具有分类特性的列构造一类包含
查询,而本文运用归纳逻辑编程算法,能够从重叠实例数据中学习出多种复杂映射,覆盖面更
广。另外,数据库领域中的模式映射\cite{30}和语义网领域中的本体映射\cite{31}已经有
不少有借鉴意义的相关研究,也有工作试图将数据库模式转换为本体后再寻找映射
\cite{28,29},但是由于关系数据库模式和本体之间不存在完美的兼容关系,所以这种转换
通常是不完备的,造成了后续本体映射的精度损失。

